\documentclass{article}
\usepackage[a4paper, total={8.5in, 11in}]{geometry}
\usepackage{csvsimple}
\usepackage{tikz}
\usetikzlibrary{calendar}
\usepackage{pgfcalendar}
\usepackage{expl3}

% sample csv file
\begin{filecontents*}{cal1.csv}
2020-01-01;2020-01-10;2020-01-03
2020-01-02;2020-01-10;2020-01-05;2020-01-25
2020-01-06;2020-01-16;2020-01-03
2020-01-08;2020-01-09;2020-01-23
\end{filecontents*}

\newlength{\daywidth}
\setlength{\daywidth}{5.0ex} %size of box around day
\newlength{\dayinterval}
\setlength{\dayinterval}{6.3ex} %space between days
\newlength{\dayitemheight}
\setlength{\dayitemheight}{0.8ex}
\newlength{\dayiteminterval}
\setlength{\dayiteminterval}{0.2ex}
\newlength{\monthwidth}
\setlength{\monthwidth}{45ex}

\ExplSyntaxOn

\seq_new:N \g_doc_dates_seq

\cs_set:Npn \doc_read_file:n #1 {
  \ior_open:Nn \g_tmpa_ior {#1}
  \seq_gclear:N \g_doc_dates_seq
  \ior_str_map_inline:Nn \g_tmpa_ior {
    \str_set:Nx \l_tmpa_str {\tl_trim_spaces:n {##1}}
    \str_if_empty:NF \l_tmpa_str {
      \seq_gput_right:NV \g_doc_dates_seq \l_tmpa_str
    }
  }
  \ior_close:N \g_tmpa_ior
}


\cs_set:Npn \doc_draw_command:nnn #1#2#3 {
  \node[#1,#2] at (#3.north) {};
}

\cs_generate_variant:Nn \regex_split:nnN {nVN}
\cs_generate_variant:Nn \doc_draw_command:nnn {xnx}

\newcommand{\drawfilerow}[3]{
  \iow_term:x {show: \fp_eval:n {-(#2 + 0.5) * (1.0ex)}pt}
  \int_compare:nT {1 <= #2 <= \seq_count:N \g_doc_dates_seq} {
    \tl_set:Nx \l_tmpa_tl {\seq_item:Nn \g_doc_dates_seq {#2}}
    \regex_split:nVN {;} \l_tmpa_tl \l_tmpa_seq
    \seq_map_inline:Nn \l_tmpa_seq {
      \str_set:Nx \l_tmpa_str {\tl_trim_spaces:n {##1}}
      \str_if_empty:NF \l_tmpa_str {
        \doc_draw_command:xnx {
          minimum~width=0.98\daywidth,
          minimum~height=\dayitemheight,
          yshift=\fp_eval:n {-(#2 - 1) * (\dayitemheight + \dayiteminterval) -0.2pt}pt,
          anchor=north,
          inner~sep=0pt,
          outer~sep=0pt
        } {fill=#3} {#1-\str_use:N \l_tmpa_str};
      }
    }
  }
}

\newcommand{\readcalendarfile}[1]{
  \doc_read_file:n {#1}
}

\ExplSyntaxOff

\begin{document}

% read "csv" file
\readcalendarfile{cal.csv}

\begin{tikzpicture}[every calendar/.style={
                    week list, 
                    month label above centered, 
                    day xshift=\dayinterval, 
                    day yshift=\dayinterval,
                    day code = { 
                        \node[minimum width=\daywidth, 
                        minimum height=\daywidth,
                        name=\pgfcalendarsuggestedname,
                        draw=black] {};
                    }
                    }]
	\calendar (mycal) [dates=2020-01-01 to 2020-01-last];
	\calendar (mycal) [dates=2020-02-01 to 2020-02-last, xshift=\monthwidth];
	\calendar (mycal) [dates=2020-03-01 to 2020-03-last, xshift=2*\monthwidth];

	\calendar (mycal) [dates=2020-04-01 to 2020-04-last, yshift=-\monthwidth];
	\calendar (mycal) [dates=2020-05-01 to 2020-05-last, xshift=\monthwidth, yshift=-\monthwidth];
	\calendar (mycal) [dates=2020-06-01 to 2020-06-last, xshift=2*\monthwidth, yshift=-\monthwidth];

   	\calendar (mycal) [dates=2020-07-01 to 2020-07-last, yshift=-2*\monthwidth];
	\calendar (mycal) [dates=2020-08-01 to 2020-08-last, xshift=\monthwidth, yshift=-2*\monthwidth];
	\calendar (mycal) [dates=2020-09-01 to 2020-09-last, xshift=2*\monthwidth, yshift=-2*\monthwidth];

   	\calendar (mycal) [dates=2020-10-01 to 2020-10-last, yshift=-3*\monthwidth];
	\calendar (mycal) [dates=2020-11-01 to 2020-11-last, xshift=\monthwidth, yshift=-3*\monthwidth];
	\calendar (mycal) [dates=2020-12-01 to 2020-12-last, xshift=2*\monthwidth, yshift=-3*\monthwidth];
 %use contents from "csv" file
    \drawfilerow{mycal}{1}{red}; %cardio
    \drawfilerow{mycal}{2}{blue}; %strength
    \drawfilerow{mycal}{3}{green}; %rope jumping
    \drawfilerow{mycal}{4}{orange}; %bouldering
    \drawfilerow{mycal}{5}{brown}; %other
\end{tikzpicture}


\end{document}

